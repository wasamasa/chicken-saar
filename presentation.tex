% Created 2019-04-06 Sat 17:31
% Intended LaTeX compiler: pdflatex
\documentclass[presentation]{beamer}
\usepackage[utf8]{inputenc}
\usepackage[T1]{fontenc}
\usepackage{graphicx}
\usepackage{grffile}
\usepackage{longtable}
\usepackage{wrapfig}
\usepackage{rotating}
\usepackage[normalem]{ulem}
\usepackage{amsmath}
\usepackage{textcomp}
\usepackage{amssymb}
\usepackage{capt-of}
\usepackage{hyperref}
\usepackage{tabu}
\usepackage{minted}
\usepackage[english]{babel}
\hypersetup{pdfauthor="Vasilij Schneidermann", pdftitle="State of Retro Gaming in Emacs", colorlinks, linkcolor=, urlcolor=blue}
\setminted{fontsize=\footnotesize,escapeinside=||}
\usetheme{Rochester}
\usecolortheme[RGB={87,83,170}]{structure}
\author{Vasilij Schneidermann}
\date{April 2019}
\title{State of Retro Gaming in Emacs}
\hypersetup{
 pdfauthor={Vasilij Schneidermann},
 pdftitle={State of Retro Gaming in Emacs},
 pdfkeywords={},
 pdfsubject={},
 pdfcreator={Emacs 26.1 (Org mode 9.1.9)}, 
 pdflang={English}}
\begin{document}

\maketitle
\begin{frame}{Outline}
\tableofcontents
\end{frame}

\AtBeginSection{\frame{\sectionpage}}

\section{Intro}
\label{sec:org95c154a}

\begin{frame}[label={sec:orgaeac524}]{About}
\begin{itemize}
\item Vasilij Schneidermann, 26
\item Software developer at bevuta IT, Cologne
\item mail@vasilij.de
\item \url{https://github.com/wasamasa}
\item \url{http://emacshorrors.com/}
\item \url{http://emacsninja.com/}
\end{itemize}
\end{frame}

\begin{frame}[label={sec:orgd02fa68}]{Motivation}
\begin{itemize}
\item Emacs is the ultimate procrastination machine
\item Many fun demonstrations:
\begin{itemize}
\item Order salad online
\item Window manager
\item IRC bot
\item Textual web browser
\item Basic games
\item 3D maze
\item Z-Machine emulator
\item Audio/video editor
\item Sex toy controller
\end{itemize}
\item Can we emulate retro games at 60 FPS?
\end{itemize}
\end{frame}

\begin{frame}[label={sec:org83df2d0}]{Context}
\begin{itemize}
\item Prior art at FrOSCon/Quasiconf: Audiovisual demonstrations
\item NES emulators are supposed to be simple
\item Random Japanese guy beat me to the punch
\item Recommended emulation project: CHIP-8
\item Alternative: Intel 8080 running Space Invaders or CP/M
\item Then someone else released a GB emulator\ldots{}
\end{itemize}
\end{frame}

\section{Interactive demonstrations}
\label{sec:org3894488}

\begin{frame}[label={sec:org57a0a33}]{NES}
\begin{itemize}
\item \url{https://github.com/gongo/emacs-nes}
\item Super slow (100x), doesn't go beyond initial game screen
\item Most time spent in rendering
\item Could maybe be made to work at acceptable speed with lots of
frameskip?
\end{itemize}
\end{frame}

\begin{frame}[label={sec:orgdd660d9}]{GB}
\begin{itemize}
\item \url{https://github.com/vreeze/eboy}
\item WIP, released in a hurry after I released mine
\item Almost playable thanks to lots of frameskip
\item Only Tetris works
\item The most popular
\end{itemize}
\end{frame}

\begin{frame}[label={sec:orgec8c7ef}]{CHIP-8}
\begin{itemize}
\item \url{https://github.com/wasamasa/chip8.el}
\item Pretty much finished, <1000SLOC
\item Supports Super CHIP-8 extensions
\item Runs at full speed, games behave OK
\end{itemize}
\end{frame}

\section{Fun facts about \texttt{chip8.el}}
\label{sec:orgb7bf18e}

\begin{frame}[label={sec:org19e92bd}]{What the hell is a CHIP-8 anyway?}
\begin{itemize}
\item It's a VM, not a console
\item Designed for easy porting of home computer games
\item Not terribly successful
\item Small community of enthusiasts writing games for it
\item There are even a few demos!
\end{itemize}
\end{frame}

\begin{frame}[label={sec:orgf2c8914}]{System specs}
\begin{itemize}
\item CPU: 8-Bit, 16 general-purpose registers, 36 instructions, each two
bytes large
\item RAM: 4KB
\item Stack: 16 return addresses
\item Resolution: 64 x 32 black/white pixels
\item Rendering: Sprites are drawn in XOR mode
\item Sound: Monotone buzzer
\item Input: Hexadecimal keypad
\end{itemize}
\end{frame}

\begin{frame}[label={sec:org0bcc3e8}]{Goals}
\begin{itemize}
\item Coming up with a name
\item Obtaining a ROM pack
\item Understanding the system
\item Basic RE tools
\item Rendering
\item Beeps
\item Make as many games run as possible
\item No debugger
\end{itemize}
\end{frame}

\begin{frame}[fragile,label={sec:org1d404e3}]{How does it work?}
 \begin{itemize}
\item Runs at an unspecified speed
\item Sound and delay timer count down at 60FPS
\item Game is loaded up at \texttt{\#x200} into RAM
\item Program counter is set to \texttt{\#x200}
\item Decode instruction, execute, loop
\end{itemize}
\end{frame}

\begin{frame}[label={sec:org30dd25d}]{Game loop woes}
\begin{itemize}
\item Game approach: Do stuff, wait, repeat
\item Doesn't work terribly well in Emacs due to user input
\item Interruptible sleep: Unpredictable
\item Un-interruptable sleep: Freezes
\item Timers: Inversion of control, allows user input to happen
\item Call a timer function at 60FPS, don't do too much in it:
\begin{itemize}
\item Execute CPU cycle(s)
\item Decrement sound/delay registers
\item Repaint
\end{itemize}
\end{itemize}
\end{frame}

\begin{frame}[label={sec:orgc5fea0b}]{Mapping the system to Emacs Lisp}
\begin{itemize}
\item It's all integers and vectors (of integers)
\item RAM, registers, return stack, key state, screen, etc.
\item Stored in global variables
\item No lists are used at all
\item Side effect: No consing happens, no GC pauses
\item Registers are mapped to a vector with an enum macro
\item Side effect: Much easier decoding
\end{itemize}
\end{frame}

\begin{frame}[label={sec:orgac95272}]{Built-in sprites}
\begin{itemize}
\item Unspecified
\item Everyone steals them from the canonical implementation
\item Super CHIP-8 has bigger sprites
\item I upscaled the small ones using a terrible Ruby oneliner
\item Lesson here: Sometimes it's not worth being clever
\end{itemize}
\end{frame}

\begin{frame}[fragile,label={sec:org467ca91}]{Decoding instructions}
 \begin{itemize}
\item All instructions are two bytes
\item Arguments are encoded inside them
\item \texttt{JP nnn} for example maps to \texttt{\#x1nnn}
\item Type extracted by masking with \texttt{\#xF000}, then shifting by 12 bits
\item Argument by masking with \texttt{\#x0FFF} (no shift needed)
\item Common patterns emerge, like addresses being the last three nibbles
\item Big \texttt{cond} dispatching on the type and executing side effects
\item Common side effect: Bumping program counter by two
\end{itemize}
\end{frame}

\begin{frame}[label={sec:orgf0cb306}]{Interactive Testing}
\begin{itemize}
\item Initially: Execute ROM until user interrupt
\item Use a debug command to render screen to a buffer
\item Maze: Small ROM, few instructions
\item There are many more ROMs that just display a static screen
\item I went through them all and added instructions as needed
\end{itemize}
\end{frame}

\begin{frame}[label={sec:orgad9ca80}]{Debugging}
\begin{itemize}
\item My usual approach of using edebug was ineffective
\item Therefore: Logging it is
\item I compared my log output with an instrumented version of evhan's
chick-8 emulator
\item If the logs diverge, that's where the bug lies
\item Future project idea: A CHIP-8 debugger, game development environment
\item Inspirations:
\begin{itemize}
\item \url{https://massung.github.io/CHIP-8/}
\item \url{http://johnearnest.github.io/Octo/}
\end{itemize}
\end{itemize}
\end{frame}

\begin{frame}[label={sec:orgcb068dc}]{Analysis}
\begin{itemize}
\item Writing a disassembler is simple, but tedious
\item Adding analysis functionality is particularly tricky
\item Idea: Reuse radare2 framework, add analysis/disasm plugin
\item I wrote one in Python, then discovered there is one in core\ldots{}
\item I then improved that one to the same level
\end{itemize}
\end{frame}

\begin{frame}[fragile,label={sec:org5f540a0}]{Unit testing}
 \begin{itemize}
\item Goal: Coverage of all instructions and what they do
\item More of a safety net, doesn't catch everything
\item Built-in ERT library isn't terribly good
\item \url{https://github.com/jorgenschaefer/emacs-buttercup} is better
\item Each test initializes the VM, loads up code, executes the
\texttt{chip8-cycle} function, checks for side effects
\end{itemize}
\end{frame}

\begin{frame}[label={sec:org7ccc69e}]{Rendering}
\begin{itemize}
\item By far the trickiest part
\item I intentionally decided against using a library
\item Creating SVGs: Too expensive
\item Creating/mutating strings: Too expensive or complicated
\item Changing SVG tiles: Gaps between lines
\item Bool vector backed XPM: Caching effects ruin everything
\item Plain text with background color: Perfect
\end{itemize}
\end{frame}

\begin{frame}[label={sec:org2f0cc04}]{Rendering optimization}
\begin{itemize}
\item Initially: Clear buffer, insert text
\item Better: Move across text, delete and insert changed parts
\item Optimization: Track dirty frame
\item Changed parts: Diff two framebuffers
\item Final optimization: Erasing text was slow, changing background text
property was way faster
\item Future optimization: Make a C module with a fast canvas
\end{itemize}
\end{frame}

\begin{frame}[fragile,label={sec:org5a25466}]{Garbage collection}
 \begin{itemize}
\item Occasionally there was a small stutter
\item This turned out to be code duplicating vectors
\item Solution: Writing a \texttt{memcpy}-style function
\item Delays after every few tests
\item Solution: Using a \texttt{memset}-like function instead of recreating
vectors
\item Hard to profile and spot, may require a custom package
\end{itemize}
\end{frame}

\begin{frame}[fragile,label={sec:orgb599a7a}]{Sound}
 \begin{itemize}
\item You only need a beep, so no difficulties emulating it
\item Playing it is hard because Emacs only supports synchronous
playback\ldots{}
\item Emacs processes are asynchronous, so controlling one works
\item \texttt{mplayer} has a slave mode, \texttt{mpv} supports listening on a FIFO for commands
\item Proof of concept:
\begin{itemize}
\item Start paused \texttt{mpv} with a FIFO in loop mode
\item Send pause/unpause command to the FIFO
\end{itemize}
\end{itemize}
\end{frame}

\begin{frame}[label={sec:org5cda192}]{User input (non-blocking)}
\begin{itemize}
\item Checking for key press state: Unsupported
\item Solution: Global key handler stores key press timestamp
\item Compare the timestamp with current time against timeout
\item Key considered pressed if less than timeout
\item Requires tweaking to feel "natural"
\end{itemize}
\end{frame}

\begin{frame}[label={sec:org17f6fd4}]{User input (blocking)}
\begin{itemize}
\item Tricky due to inversion of control
\item Required me to do a state machine rewrite
\item The command transfers the emulator into a waiting state
\item The global key handler checks for that state and transfers to the
playing state
\end{itemize}
\end{frame}

\begin{frame}[label={sec:org40c51e0}]{Super CHIP-8}
\begin{itemize}
\item Supports more interesting games
\item Proper scrolling support requires tricks to do in-place
\item Doubled resolution required an extra rendering optimization
\item It's possible to switch between both modes, making it tricky to
implement:
\begin{itemize}
\item You could always work in high-res and downscale if needed
\item Alternatively: Switch between low-res and high-res screen to
render to
\item I went for the latter
\end{itemize}
\end{itemize}
\end{frame}

\begin{frame}[label={sec:orga1710c3}]{Other stuff}
\begin{itemize}
\item Sometimes games deviate from the reference, conflicting with it
\item Sometimes it's unclear whether it's worth it to support an obscure
feature
\item I'm not good at games and didn't enjoy playing them
\item However: You gain great insight how the machine works
\end{itemize}
\end{frame}

\section{Outro}
\label{sec:org734f4e8}

\begin{frame}[label={sec:org9dc929e}]{What next?}
\begin{itemize}
\item Maybe an Intel 8080 emulator running CP/M
\item Maybe experimentation with faster rendering
\item More serious stuff in CHICKEN, like NES or GB emulator
\end{itemize}
\end{frame}

\begin{frame}[label={sec:org112ed00}]{Questions?}
\end{frame}
\end{document}